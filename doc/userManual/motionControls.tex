\subsection{Computation methods}
\label{sss:aquagpusph:motions:Controls}
%
\begin{center}
\textbf{Linear interpolated data table}
\end{center}
%
The simplest method to set the motion data is providing a tabulated file with the required data (see section
\ref{sss:aquagpusph:motions:Types} to know what data is required by the selected motion). Tabulated file
must contain one line per time instant with time at first column, and all the request fields by motion at the next
ones.\\
%
Comments can be included in the file with the symbol '\#'. All the line content after '\#' symbol will be
ignored. Fields can be separated by comma, semicolon, parenthesis, spaces or tabulator symbols, and if several
separators are concatenated between the field values then will be joined as a unique space separator.\\
%
When values are requested for a time instant $t$, such that $t_n < t < t_{n+1}$, values will be linearly interpolated
%
\[x(t) = \frac{t - t_n}{t_{n+1}-t_n} x(t_{n+1}) + \left( 1 - \frac{t - t_n}{t_{n+1}-t_n} \right) x(t_n)\]
%
You can see the section \ref{s:examples} to see practical application of linear interpolated tabulated motions data.
%
\begin{center}
\textbf{Python script controlled motion}
\end{center}
%
The most flexible and powerful method to control the motion is using a Python script where 2 methods must be
implemented:\\
%
\underline{\textbf{init}}\\
%
Method used to give the needed fields, as described on section \ref{sss:aquagpusph:motions:Types} for each type of
motion, for the time instant $t = 0 \mathrm{s}$, that can be useful to set initial condition.\\
%
\underline{\textbf{perform}}\\
%
Method called each time step in order to get all needed fields, as described on section
\ref{sss:aquagpusph:motions:Types} for each type of motion.
%
Input and output arguments may vary for both methods depending on the type of motion selected. We really encourage
user to go to chapter \ref{s:examples}, where practical applications of Python controlled motions can be found.
%