\subsubsection{$[$ERROR$]$ VTK output called, but not supported:}
%
If \NAME has been built without VTK support, but VTK files output have
been requested, this error will be reported each time that output is
called. See chapter \ref{sss:install:cmake} in order to know how to
activate support for VTK files.
%
\subsubsection{$[$ERROR$]$ H5Part output called, but not supported:}
%
If \NAME has been built without H5Part support, but H5Part files output
have been requested, this error will be reported each time that output
is called. See chapter \ref{sss:install:cmake} in order to know how to
activate support for H5Part files.
%
\subsubsection{$[$ERROR$]$ Can't send variable to kernel:}
%
Error produced when \NAME can't send a variable to an OpenCL kernel. If
you modified an OpenCL kernel and/or the source code to customize \NAME
functionality, ensure that sent variables from \NAME and declared ones
on OpenCL kernel matchs.\rc
This error will stop \NAME execution.
%
\subsubsection{$[$ERROR$]$ Can't execute the kernel:}
%
OpenCL kernel can't be launched. If the problem is specifically documented,
it will be added to the log too. Common causes of this problem is a custom
OpenCL kernel, or unallocatable resources.\rc
This error will stop the \NAME execution.
%
\subsubsection{$[$ERROR$]$ Can't wait to kernels end:}
%
If \NAME has been built in Debug mode, an execution profile will be activated in
order to report time consumed by each stage of the code. This error is reported when
program can't wait for kernel execution, usually caused by a kernel execution error.\rc
This error will stop the \NAME execution.
%
\subsubsection{$[$ERROR$]$ Can't profile kernel execution:}
%
Similar to previous one, in this case program can wait to kernel finish, but can not be
profiled (execution time can not be extracted).\rc
This error will stop the \NAME execution.
%
\subsubsection{$[$ERROR$]$ Resultant keys overflows unsigned int type:}
%
Radix sort stage related error. This error is usually caused by a too large number
of cells, that can happen if a domain is not bounded (see section \ref{sss:XML:SPH} in order t
know how to set it) and one or more particles go out of physical boundaries.\rc
This error will stop the \NAME execution.
%
\subsubsection{$[$ERROR$]$ perform() execution fail:}
%
Error associated to simulations with Python script motion control. The error is
caused by a Python runtime execution error detected at perform() required method.
Motion control Python script must be fixed.\rc
Check chapter \ref{sss:XML:Movements} in order to learn more about how to set a Python script
controlled motion, and \ref{sss:aquagpusph:motions:Controls} to learn more about the Python scripting.\rc 
This error will stop the \NAME execution.
%
\subsubsection{$[$ERROR$]$ perform() returned quaternion is not valid:}
%
Error associated to simulations with Python script motion control. The error is
caused by a wrong returned value by Python script perform() method.
Motion control Python script must be fixed.\rc
You can see chapter \ref{sss:XML:Movements} in order to learn more about how to set a Python script
controlled motion, and \ref{sss:aquagpusph:motions:Controls} to learn more about the Python scripting.\rc 
This error will stop the \NAME execution.
%
\subsubsection{$[$ERROR$]$ Failure retrieving memory from server:}
%
OpenCL memory can not be recovered on host. \NAME eventually provide details
of this error into the log.\rc
This error will stop the \NAME execution.
%
\subsubsection{$[$ERROR$]$ Failure sending memory to server:}
%
Data can not be sent from host to OpenCL platform. \NAME eventually provide details
of this error into the log.\rc
This error will stop the \NAME execution.
%
\subsubsection{$[$ERROR$]$ Fail allocating memory for $array$ ($m$ bytes):}
%
Produced by an unacceptable number of cells number. Requested memory on OpenCL
platform can't be allocated.\rc
This error will stop the \NAME execution.
%
\subsubsection{$[$ERROR$]$ timestep has dramaticaly decreased! [$dt_t$ -$>$ $dt_{t+dt}$]:}
%
In \NAME a variable time step can be set, see chapter \ref{sss:XML:Timing} to learn more
about the time step alternatives. Variable time step is a good way to walk around
some transitional instabilities, decreasing instantaneously the time step,
and trying to avoid that the particles pass trough walls therefore.\rc
This error is reported when the time step is reduced in one or more orders of
magnitude. If this error is reported the simulation can continue, but long
computational times with wrong results can be expected, and you may
consider the possibility of stop and fix simulation in order to relaunch it.
%