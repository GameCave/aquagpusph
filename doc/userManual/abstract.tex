\chapter{Introduction}
\label{s:introduction}
%
\section{About \NAME}
%
\NAME (Another QUAlity GPU-SPH) is a free CFD software licensed under GPLv3,
and developed by J.L. Cercos-Pita as part of his PhD at CEHINAV-UPM group
\footnote{Model Basin Research group, Technical University of Madrid (UPM),
\url{http://canal.etsin.upm.es}}. \NAME is based on Lagragian meshfree SPH
(Smoothed Particles Hydrodynamics) method.\rc
%
\NAME has been accelerated with OpenCL, that allows you to execute it over
CPU based platforms, over GPUs based platforms, and eventually, over every platform
developed in the future adapted to the OpenCL standard. For the moment the simultaneous
use of several platforms in the same simulation is not supported.\rc
%
The code has been widely validated, used and documented in several publications, for
instance \citet{Maciaetal_PTP_2012} where boundary integrals are tested or
\citet{perezrojas_cercos_stab12} where SPHERIC benchmark test case number 9 was simulated.
Some examples provided within the package have been studied in these publications.\rc
%
\NAME has been designed for UNIX like operative system, and tested on GNU/Debian Linux
distributions. For the moment no ports has been developed for Windows or Mac operative
systems.
%
\NAME has been developed in C\texttt{++} and OpenCL\footnote{OpenCL is so quite similar
to C}. Also Python extensions has been developed.
%
\section{License notes}
%
\NAME is free software: you can redistribute it and/or modify
it under the terms of the GNU General Public License as published by
the Free Software Foundation, either version 3 of the License, or
(at your option) any later version.\rc
%
\NAME is distributed in the hope that it will be useful,
but WITHOUT ANY WARRANTY; without even the implied warranty of
MERCHANTABILITY or FITNESS FOR A PARTICULAR PURPOSE.  See the
GNU General Public License for more details.\rc
%
You should have received a copy of the GNU General Public License
in 'LICENSE' file with \NAME package. If not, see
\url{http://www.gnu.org/licenses/}.
%
\section{Objectives of the present document}
%
This document has been created aiming to provide a tool that allows the \NAME users to know:
%
\begin{enumerate}
	\item What is \NAME program.
	\item What you can solve with \NAME software.
	\item How exactly is \NAME performing the simulations.
	\begin{enumerate}
		\item Physical and numerical model.
		\item Algorithmic used.
	\end{enumerate}
	\item How you can setup your simulation and run it.
	\item How you can post-process the results.
\end{enumerate}
%
However, this document can complement the developers documentation as well, composed mainly
by the Doxygen documentation.\rc
%
\textcolor{red}{This document is currently in process, and therefore some 
chapters may be outdated.
%
It is strongly recommended to build the examples in order to know how to setup 
new simulations.}
%
\section{Required background}
%
Like many other CFD\footnote{Computational Fluid Dynamics}, the usage of the program requires
a minimum background on fluid mechanics\footnote{And the involved mathematics} in order to
understand it.\rc
%
At the other hand, in order to understand all the possibilities that \NAME offers to you, SPH
background is strongly recommended. Nevertheless the SPH method, and more specifically the
model internally developed in \NAME CFD, is detailed described later.
%
\section{Organization of this document}
%
The document starts with an introduction in the chapter \ref{s:introduction} where \NAME
software is presented, and the present document described.\rc
%
Then the physical and numerical model internally developed in \NAME is described in chapter
\ref{s:model}. The model introduced contains all the possibilities, including corrections
or boundary conditions, but should be the user who's select what of them is applied.\rc
%
In the chapter \ref{s:install} the \NAME install process is documented.\rc
%
All the options, inputs and commands in order to setup, run, and post-process simulations is
documented in chapters \ref{s:caseSetup}, \ref{s:running} and \ref{s:outputFiles}. The
objective of these chapters is to provide a reference in order to users can know where and
how exactly different options, commands or inputs is used, but is not designed as practical
section, that is rely to chapter \ref{s:examples} about the examples provided with \NAME
package.\rc
%
Algorithmic topics are described in the chapter \ref{s:aquagpusph}, where general information
about how \NAME works internally is documented. For further details you can access to the
Doxygen documentation, and if you need to go in depth you can ever read the source code.
%