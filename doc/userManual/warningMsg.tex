\subsubsection{$[$WARNING$]$ timestep changed [$dt_t$ -$>$ $dt_{t+dt}$]:}
%
In \NAME a variable time step can be set, see chapter \ref{sss:XML:Timing} to learn more
about the time step alternatives. Variable time step is a good way to walk around
some transitional instabilities, decreasing instantaneously the time step, and
trying to avoid that the particles pass trough walls therefore.\rc
This message indicates that the time step has been adapted. The time step variation
is an indicative that the simulation is not running right, so if you receive
too much time step change reports, or the time step change is large, maybe you
must consider repeat the simulation increasing the sound speed or the time step
divisor.
%
\subsubsection{$[$WARNING$]$ Number of cells increased [$n_t$ -$>$ $n_{t+dt}$]:}
%
Like other SPH codes, \NAME neighbour particles localization is based on a
PIC algorithm, where each particle is marked by a grid cell where is situated. The
number of cells where the particles can be located depends on the extension of
fluid domain. Therefore, it can be modified along the simulation. You can learn more about
cells usage on \NAME at chapter \ref{ss:aquagpusph:linklist}.\rc
In most simulations fluid bounds are not constant along the simulation, so some
reports about cell numbers increase can occur. If this increase is too large the reallocation of
memory can be impossible, breaking the simulation, and possibly indicating that some
particles has abandoned the physical domain.
%
