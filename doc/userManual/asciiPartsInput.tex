\subsection{ASCII formatted file}
\label{sss:partsFile:ASCII}
%
It is probably the best option in the most cases, and has
been the option selected for the examples presented in the
chapter \ref{s:examples}. Is a plain text file where all
particles are defined each one in a file line.\rc
%
Comments can be included in the file with the symbol '\#'.
All the line contents after the '\#' symbol will be ignored.\rc
%
\NAME expects that following fields per particle are present
on the provided files:
%
\begin{enumerate}
	\item \textbf{$x$} (Mandatory): X coordinate $[\mbox{m}]$.
	\item \textbf{$y$} (Mandatory): Y coordinate $[\mbox{m}]$.
	\item \textbf{$z$} (Mandatory, only for 3D): Z coordinate $[\mbox{m}]$.
	\item \textbf{$n_x$} (Mandatory): X normal component. Fluid particles can have null normal.
	\item \textbf{$n_y$} (Mandatory): Y normal component. Fluid particles can have null normal.
	\item \textbf{$n_z$} (Mandatory, only for 3D): Z normal component. Fluid particles can have null normal.
	\item \textbf{$v_x$} (Mandatory): X velocity component $[\mbox{m/s}]$.
	\item \textbf{$v_y$} (Mandatory): Y velocity component $[\mbox{m/s}]$.
	\item \textbf{$v_z$} (Mandatory, only for 3D): Z velocity component $[\mbox{m/s}]$.
	\item \textbf{$m$} (Mandatory): Mass if it is a fluid or fix particle $[\mbox{kg}]$, area if it is a wall element $[\mbox{m}^2]$.
	\item \textbf{$imove$} (Optional): Moving flag. $imove > 0$ for all fluid particles, $imove < 0$ for fixed
	particles or vertexes ($imove = 0$ is reserved for sensors).
	\item \textbf{$\rho$} (Optional): Density $[\mbox{kg/m}^3]$.
	\item \textbf{$c_S$} (Optional): Sound speed $[\mbox{m/s}]$. Sound speed must be selected, at least, 10
	times greater than maximum expected fluid velocity.
	\item \textbf{$h$} (Optional): Kernel characteristic height $[\mbox{m}]$.
\end{enumerate}
%
Field values must be set sorted, and can be separated by comma,
semicolon, parenthesis, spaces or tabulator symbols. If several
separators are concatenated between the field values will be
joined as a space separator.\rc
%
If one mandatory field is missing \NAME will report an error on
the initialization stage. Regarding optional fields,
it's strongly recommended set $imove$ flag and density, that will
be set as $imove = 1$ and $\rho = \rho_0$ by default. Be mindful
that the state equation relates pressure and density fields, so if
you want to perform a simulation with some initial pressure field
you need to setup the density field according to the EOS
\ref{eq:governing_eqns:field_eqns:eos:arfm_2012}.\rc
%
Mass fields is a bit particular variable since if the particle is
a boundary area element, this field must be set as the area
assigned to it.\rc
%
Take care also with the 2D simulations since the field values are
described per meter of depth (3D missing direction). For instance
area of the wall elements will be the line length $[\mbox{m}]$.
You can see the provided examples in order to know more about this.\rc
%
When a file with an unknown extension is provided in order to load
the particles, it will be treated as ASCII file type.
%
