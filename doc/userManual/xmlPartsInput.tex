\subsection{XML formatted file}
\label{sss:partsFile:XML}
%
This is an auxiliar method that is only recommended for a really
low number of particles. In this method an XML file is requested,
similar to the explained in chapter \ref{s:caseSetup}, but
expecting \textbf{Particle} group instances (one per particle).\rc
%
In table \ref{tables:caseSetup:PartInput:Options} a list with the
valid tags and their attributes for each particle are related. The
configurable fields are similar to the introduced at
\ref{sss:partsFile:ASCII} section about the ASCII formatted input;
revisit this section in order to know the possibilities of initial
configuration file.
%
\begin{table}[h!b!p!]\small
	\centering
	\begin{tabular}{| c | c | c | l | }
		\hline
		\cellcolor[rgb]{0.7,0.7,0.7}Tag & \cellcolor[rgb]{0.7,0.7,0.7}Attributes & \cellcolor[rgb]{0.7,0.7,0.7}Description \\
		\hline
		Position & x     & (Mandatory) X coordinate $[\mbox{m}]$. \\
		         & y     & (Mandatory) Y coordinate $[\mbox{m}]$. \\
		         & z     & (Mandatory, only for 3D) Z coordinate $[\mbox{m}]$. \\
		\hline
		Normal   & x     & (Mandatory) X normal component. \\
		         & y     & (Mandatory) Y normal component. \\
		         & z     & (Mandatory, only for 3D) Z normal component. \\
		\hline
		Velocity & x     & (Mandatory) X velocity component $[\mbox{m/s}]$. \\
		         & y     & (Mandatory) Y velocity component $[\mbox{m/s}]$. \\
		         & z     & (Mandatory, only for 3D) Z velocity component $[\mbox{m/s}]$. \\
		\hline
		Mass     & value & (Mandatory) Mass for fluid or fix particles $[\mbox{kg}]$, \\
		         &       & area for vertexes $[\mbox{m}^2]$. \\
		\hline
		Imove    & value & (Optional) Moving flag. \\
		\hline
		Density  & value & (Optional) Density $[\mbox{kg/m}^3]$. \\
		\hline
		Cs       & value & (Optional) Sound speed $[\mbox{m/s}]$. \\
		\hline
		KernelH  & value & (Optional) Kernel characteristic height $[\mbox{m}]$. \\
		\hline
	\end{tabular}
	\caption{Particles valid fields tags.}
	\label{tables:caseSetup:PartInput:Options}
\end{table}
%
