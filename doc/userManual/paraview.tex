\section{Visualization files}
\label{ss:running:visualizationoutput}
%
\subsection{General}
%
\NAME actually offers 2 types of visualization output files, Tecplot and H5Part
ones. Tecplot output files are outdated, and since is a plain text output format,
is created and loaded slowly and hight hard disk consuming, so is strongly
recommended to don't use this format. However H5Part has been spicifically designed
to perform particles output in a binary form, being computationally efficient
compared with other formats, and is the format that will be documented here.
%
\NAME VTK format support is incoming, but not implemented yet.
%
\subsection{H5Part}
\label{sss:visualizationoutput:H5Part}
%
H5Part is an high efficient implementation of \href{http://www.hdfgroup.org/HDF5}{hdf5}
file template oriented to particles based output. Main advantages are:
%
\begin{enumerate}
	\item Is a binary format that reduce significantly the Hard Disk resources consuming.
	\item Is a free library.
	\item All time steps can be packed on one file.
\end{enumerate}
%
And main disadvantages are:
%
\begin{enumerate}
	\item Is not generalized format, so is not commonly introduced on standard package repository
	of most usually used Linux distributions.
	\item If program fails before file is right closed, will result on unrecoverable data.
\end{enumerate}
%
On actual \NAME implementation this is the main output format. H5Part files can be postprocessed
with ParaView\footnote{ParaView H5Part Reader plugin must be loaded}. This chapter is not
oriented to describe the work with ParaView, please refer to ParaView's web page to find tutorials,
in this chapter we will describe some specific operations that can helps you processing data.\rc
%
In \NAME particles have 2 relevant flags, \textit{\textbf{imove}} and \textit{\textbf{ifluid}}. The
first one marks if the particle is:
%
\begin{enumerate}
	\item $imove < 0$ a fix particle or a wall vertex.
	\item $imove = 0$ a sensor.
	\item $imove > 0$ a fluid particle.
\end{enumerate}
%
So you can use \textbf{Clip} tool over this scalar in order to split particles in this 3 categories
that can be visualized independently. Regarding \textit{\textbf{ifluid}} flag, marks the fluid of
each particle, allowing you extract different fluids in order to process it independently of the 
other ones, with the \textbf{Clip} filter as well.\rc
%
All the fields exported by \NAME are scalars, including vectors components. You can build vectorial
fields for some specific purposes with the \textbf{Calculator} tool, simply set the sum of each
vector component multiplied by \textit{iHat, jHat or kHat} associated variable.\rc
%
You can get a large number of additional plugins specifically oriented to SPH visualization works
from
\href{https://hpcforge.org/plugins/mediawiki/wiki/pv-meshless/index.php/Main_Page_for_pv-meshless_WIKI}{pv-meshless}.\rc
%
On actual \NAME package is not possible to select what fields will be printed, printing all relevant
particles data. Future releases will offer the possibility of discard some fields of output in order
to decrease the output file size.
%
