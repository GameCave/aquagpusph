\section{Energy data file}
\label{ss:running:energyoutput}
%
Energy file is a standard output file of \NAME program, you can set printing rate of this file
as explained on \ref{sss:XML:Timing} section. Energy data file is called ``Energy.dat''\footnote{If
a prefix is set on command line arguments, will inserted on the start of the file name}.\rc
%
Energy data file printed by \NAME has a header that explains the fields printed into it. Following
columns are printed on energy data file:
%
\begin{enumerate}
	\item \textbf{Time}: Print time $[\mbox{s}]$.
	\item \textbf{X\_Mom}: Momentum at X direction $[\mbox{N} \cdot \mbox{s}]$.
	\item \textbf{Y\_Mom}: Momentum at Y direction $[\mbox{N} \cdot \mbox{s}]$.
	\item \textbf{Z\_Mom} (Only for 3D): Momentum at Z direction $[\mbox{N} \cdot \mbox{s}]$.
	\item \textbf{X\_AngMom} (Only for 3D): Angular momentum respect X axis $[\mbox{N} \cdot \mbox{m s}]$.
	\item \textbf{Y\_AngMom} (Only for 3D): Angular momentum respect Y axis $[\mbox{N} \cdot \mbox{m s}]$.
	\item \textbf{Z\_AngMom}: Angular momentum respect Z axis $[\mbox{N} \cdot \mbox{m s}]$.
	\item \textbf{Kinetic\_Energy}: Kinetic energy $[\mbox{J}]$.
	\item \textbf{Potential\_Energy}: Potential energy $[\mbox{J}]$.
	\item \textbf{Total\_Energy}: Total energy $[\mbox{J}]$.
\end{enumerate}
%
Each line represents a time step of the simulation, characterised by simulation time instant. Energy
data (including momentums and angular momentums) only considers fluid particles, but not vertexes,
fixed particles or sensors.\rc
%
This data output file can easily plotted with \href{http://www.gnuplot.info}{gnuplot} for instance.
%
\begin{verbatim}
#########################################################
#                                                       #
#    #    ##   #  #   #                           #     #
#   # #  #  #  #  #  # #                          #     #
#  ##### #  #  #  # #####  ##  ###  #  #  ## ###  ###   #
#  #   # #  #  #  # #   # #  # #  # #  # #   #  # #  #  #
#  #   # #  #  #  # #   # #  # #  # #  #   # #  # #  #  #
#  #   #  ## #  ##  #   #  ### ###   ### ##  ###  #  #  #
#                            # #             #          #
#                          ##  #             #          #
#                                                       #
#########################################################
#
# Another QUAlity GPU-SPH, by CEHINAV.
#	http://canal.etsin.upm.es/
# Authors:
#	Jose Luis Cercos-Pita
#	Leo Miguel Gonzalez
#	Antonio Souto-Iglesias
#
#########################################################
#
# GnuPlot script to plot energy global variables in real time.
#
#########################################################
#
set multiplot
set key
set grid
# Line styles
set style line 1 lt 1 lw 2
set style line 2 lt 2 lw 2
set style line 3 lt -1 lw 2
# Plot Kinetic & gravity energy
set xlabel "Time [s]"
set xtic
set ylabel "Kinetic Energy [J]"
set ytic
set y2label "Gravity Energy [J]"
set y2tic
set size square 0.5,1.0
set origin 0.0,0.0
plot "Energy.dat" using 1:5 title 'Kinetic energy' axes x1y1 with lines ls 1, \
     "" using 1:6 title 'Gravity energy' axes x1y2 with lines ls 2

# Plot global energy
set origin 0.5,0.0
set ylabel "Energy [J]"
unset y2label
unset y2tic
plot "Energy.dat" using 1:7 title 'Energy' axes x1y1 with lines ls 3
set nomultiplot

# Replot in 5 seconds
pause 5
replot
reread
\end{verbatim}
%
Attached script can be used to plot kinetic, potential, and total energies during
the simulation is running. The plot will be autoupdated (5 seconds refresh period),
showing Kinetic energy on the left plot, refereed to left y axis, potential energy
on the left plot as well, but refereed to right y axis, and total energy on the
right plot.\rc
%
Similar scripts can be easily make to plot momentums or angular momentums, or in
order to create images (without showing and refreshing) as well.
%
\section{Bounds data file}
\label{ss:running:boundsoutput}
%
Bounds data file is so quite similar to the previous one. You can find how to set
printing rate on section \ref{sss:XML:Timing} as well. Bounds data file is called
``Bounds.dat''\footnote{If a prefix is set on command line arguments, will inserted
on the start of the file name}.\rc
%
This file contains following data fields:
%
\begin{enumerate}
	\item \textbf{Time}: Print time $[\mbox{s}]$.
	\item \textbf{X\_Min}: Minimum X coordinate $[\mbox{m}]$.
	\item \textbf{Y\_Min}: Minimum Y coordinate $[\mbox{m}]$.
	\item \textbf{Z\_Min} (Only for 3D): Minimum Z coordinate $[\mbox{m}]$.
	\item \textbf{X\_Max}: Maximum X coordinate $[\mbox{m}]$.
	\item \textbf{Y\_Max}: Maximum Y coordinate $[\mbox{m}]$.
	\item \textbf{Z\_Max} (Only for 3D): Maximum Z coordinate $[\mbox{m}]$.
	\item \textbf{V\_Min}: Minimum velocity $[\mbox{m/s}]$.
	\item \textbf{V\_Max}: Maximum velocity $[\mbox{m/s}]$.
\end{enumerate}
%
Like energy data, in bounds data only fluid particles are considered, discarding
fixed particles, vertexes and sensors therefore.\rc
%
On previous section you have a script that can be easily fit to plot bounds data.
%
\section{Sensors data file}
\label{ss:running:sensorsoutput}
%
Sensors data is a little bit different. Sensors file will printed only if you add
at least one sensor as explained on chapter \ref{sss:XML:Sensors}, where you can set
printing rate too. Sensors data file is called ``Sensors.dat''\footnote{If a prefix
is set on command line arguments, will inserted on the start of the file name}.\rc
%
Sensors file contains following fields (explained in a header):
%
\begin{enumerate}
	\item \textbf{Time}: Print time $[\mbox{s}]$.
	\item \textbf{X} (One per sensor): X coordinate $[\mbox{m}]$.
	\item \textbf{Y} (One per sensor): Y coordinate $[\mbox{m}]$.
	\item \textbf{Z} (One per sensor, only for 3D): Z coordinate $[\mbox{m}]$.
	\item \textbf{press} (One per sensor): Pressure $[\mbox{Pa}]$.
	\item \textbf{density} (One per sensor): Density $[\mbox{kg/m}^3]$.
	\item \textbf{sumW} (One per sensor): Kernel completeness factor (formerly Shepard term).
\end{enumerate}
%
This file has the singularity that 6 last described fields are repeated as many times as sensors
have been added. Pressure and density written in file are ever renormalized using Shepard
correction, if you want to know the traditional SPH interpolated value you must multiply the values
by the Shepard term (last field of each sensor).\rc
%
Sensors data file can be easily plotted with gnuplot as well.
%