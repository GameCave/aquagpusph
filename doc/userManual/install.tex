\chapter{Install \NAME}
\label{s:install}
%
\section{General}
%
\NAME can be downloaded from
%
\begin{center}
\url{http://canal.etsin.upm.es/aquagpusph/descargas.php}
\end{center}
%
The process to configure, build and install is described below. The files
uploaded to this web page are the latest stable version, but optionally
you can download the developers version, that has more fresh features, but
can carry some bugs and unstabilities. Developers version is allocated in
\href{https://github.com/}{github}, so you must have git installed on your
system to access to the source code. To download the developers package
version execute
%
\begin{verbatim}
git clone git://github.com/jlcercos/aquagpusph.git
\end{verbatim}
%
that will generate a folder called ``aquagpusph'' with the package inside.\rc
%
The instructions included below are valid for both the stable and the
developers versions, but some additional dependencies and configuration can
be missed.\rc
%
For the moment no specific GNU/Linux distributions installable packages have
been created.
%
\section{Dependencies}
%
\NAME has some dependencies that must be installed before. Almost dependencies
are mandatory, but exist other dependencies that can be optionally ignored
depending on the needed capabilities on \NAME built program.\rc
%
Dependencies are classified in 4 categories therefore:
%
\begin{enumerate}
	\item Mandatory: \NAME needs the dependency available on system in order to
	run.
	\item Optional: If dependency is not available, \NAME can be built, but some
	features will be disabled.
	\item Recommended: If dependency is not available, \NAME can be built with
	all the features, but some
	additional package characteristics may not run.
	\item Suggested: \NAME can have some synergies with other softwares, for
	preporcessing or postprocessing for instance.
\end{enumerate}
%
In table \ref{tables:install:dependencies} you can see the list of dependencies
required. Each dependency can require more other dependencies that will not be
documented. Take into account the following considerations:
%
\begin{enumerate}
	\item OpenCL is a little bit special dependency due to is really hardware
	depending, therefore at the web page only OpenCL specification will found.
	We encourage you to contact your hardware provider in order to know the
	best way to install it.
	\item NCurses can be used to perform smart screen output, that can revert
	on slightly reduced computational time. Any relevant functionality will
	be lost if \NAME is compiled without NCurses support therefore.
	\item While H5Part was the main \NAME output format in previous versions, 
	it has been replaced by VTK files (H5Part has been declared as an outdated 
	format).
	\item Doxygen and Graphviz are only needed if you want to build Doxygen
	developers documentation. This documentation is mainly oriented to 
	developers, but can be really useful for users in some number of cases, so 
	is strongly recommended to build it.
\end{enumerate}
%
\begin{table}[h!b!p!]\small
	\centering
	\begin{tabular}{| c | c | c | }
		\hline
		\cellcolor[rgb]{0.7,0.7,0.7}Dependency &\cellcolor[rgb]{0.7,0.7,0.7}Status &\cellcolor[rgb]{0.7,0.7,0.7}Web page\\
		\hline
		CMake    & Mandatory   & \url{http://www.cmake.org} \\
		\hline
		xerces-c & Mandatory   & \url{http://xerces.apache.org/xerces-c} \\
		\hline
		Python   & Mandatory   & \url{http://www.python.org} \\
		\hline
		OpenCL   & Mandatory   & \url{http://www.khronos.org/opencl} \\
		\hline
		Eigen3   & Mandatory   & \url{http://eigen.tuxfamily.org/index.php} \\
		\hline
		libmatheval & Mandatory   & 
		\url{https://www.gnu.org/software/libmatheval} \\
		\hline
		Ncurses  & Optional    & \url{http://www.gnu.org/software/ncurses/ncurses.html} \\
		\hline
		VTK      & Optional    & \url{http://www.vtk.org} \\
		\hline
		Doxygen  & Recommended & \url{http://www.doxygen.org} \\
		\hline
		Graphviz & Recommended & \url{http://www.graphviz.org} \\
		\hline
		matplotlib & Suggested   & \url{http://matplotlib.org} \\
		\hline
		numpy & Suggested   & \url{http://www.numpy.org} \\
		\hline
		PyQt4 & Suggested   & 
		\url{http://www.riverbankcomputing.com/software/pyqt/intro} \\
		\hline
		gnuplot  & Suggested   & \url{http://www.gnuplot.info} \\
		\hline
		ParaView & Suggested   & \url{http://www.paraview.org/} \\
		\hline
	\end{tabular}
	\caption{\NAME dependencies.}
	\label{tables:install:dependencies}
\end{table}
%
\section{Install}
%
\subsection{General}
%
\NAME use \href{http://www.cmake.org}{CMake} to the configure, build and 
install process. CMake installation process is performed in 3 steps therefore:
%
\begin{enumerate}
	\item \textbf{CMake configuration}: In this stage main options and features
	that the built version of \NAME will have.
	\item \textbf{Compilation}: In this step the source code is compiled into a
	binary ready to be launched.
	\item \textbf{Install}: This step is optional, aiming to install the 
	software on the system in order to be available for all the users of the 
	computer.
\end{enumerate}
%
To perform the CMake configuration process 2 ways are described.
%
\subsection{CMake configuration}
\label{sss:install:cmake}
%
To start configuring CMake you can run the following command on the folder 
where you downloaded \NAME sources:
%
\begin{verbatim}
cmake .
\end{verbatim}
%
That will configure \NAME with default options. You can modify some options, 
for instance, to change installation prefix from $\mbox{/usr/local}$ to 
$\mbox{/usr}$ folder, and build Release version, you can launch following 
command:
%
\begin{verbatim}
cmake . -D CMAKE_INSTALL_PREFIX:PATH=/usr -D CMAKE_BUILD_TYPE:STRING=Release
\end{verbatim}
%
In \url{http://www.cmake.org/Wiki/CMake_Useful_Variables} web page you can 
find a list with the most commonly CMake used options. Also \NAME provides 
some additional options:
%
\begin{enumerate}
	\item \textbf{AQUAGPUSPH\_3D}: ON to build 3D \NAME version, OFF to build 
	2D version. \NAME 2D and 3D versions can be installed on the same system. 
	Be mindful that 3D version must not be used to perform 2D simulations.
	\item \textbf{AQUAGPUSPH\_BUILD\_DOC}: ON to build 
	\href{http://www.doxygen.org}{Doxygen} developers documentation, OFF 
	otherwise. Requires \href{http://www.doxygen.org}{Doxygen} and 
	\href{www.graphviz.org}{Graphviz} packages installed in the system.
	\item \textbf{AQUAGPUSPH\_BUILD\_EXAMPLES}: ON to generate the examples. 
	Generated examples are the associated to the selected 2D/3D version.
	\item \textbf{AQUAGPUSPH\_USE\_NCURSES}: ON to build \NAME with NCurses 
	output terminal, OFF otherwise. NCurses has more efficient screen streamed 
	functionality, so you can win some performance using it.
	\item \textbf{AQUAGPUSPH\_USE\_VTK}: ON to build \NAME with VTK output 
	support, OFF otherwise. In \NAME- \VERSION VTK is the only valid output 
	format for the particles files, so it is mandatory to can process 
	visualizations of the simulations.
\end{enumerate}
%
\subsection{CMake configuration with ccmake}
%
I order to simplify the process you can use ccmake\footnote{Or eventually 
CMake-GUI} to configure \NAME with a more friendly user interface (Of course 
you must install ccmake before). Simply launch the command:
%
\begin{verbatim}
ccmake .
\end{verbatim}
%
If you did not configure \NAME before, an empty cache page will shown up, press
\textbf{'c'} key to perform initial default configuration. After some seconds 
working, errors, warnings, and most relevant selected options report will be 
returned; press \textbf{'e'} key to continue.\rc
%
Now some options are shown in order to allow you to edit them.
%
Probably you want to edit CMAKE\_INSTALL\_PREFIX too. Take care about
CMAKE\_BUILD\_TYPE, Debug mode is a lot of more resources consuming and slow, 
therefore it's strongly recommended to let Release mode for almost users.\rc
%
After this, press \textbf{'c'} key in order to configure the package again. 
Repeat the process until no new options are shown (that are marked with an 
asterisk), and no more errors are reported.
%
At this point ccmake will offer generating final project pressing \textbf{'g'} 
key.\rc
%
Same options introduced in \ref{sss:install:cmake} can be set using ccmake.
%
\subsection{Compilation}
%
After the configuration process you can start the compilation in order to 
create the binary.
%
You can launch the following command:
%
\begin{verbatim}
make all
\end{verbatim}
%
Since \NAME build process is not really time consuming parallel compiling 
process is not generally needed, but of course you can set several processors 
building simultaneously.
%
For instance, if you want to use 8 cores you can type:
%
\begin{verbatim}
make -j8
\end{verbatim}
%
Additionally, if you want to get more info about the building process, you can 
use VERBOSE flag:
%
\begin{verbatim}
make VERBOSE=1
\end{verbatim}
%
This will report the command executed to compile/generate each file.
%
\subsection{Install}
%
\NAME can be used without installing it, see generated examples to learn more 
about this, but if \NAME must be provided to several users probably you want 
to install it typing:
%
\begin{verbatim}
make install
\end{verbatim}
%
Note that depending the selected installation folder you may need administrative permissions.
%