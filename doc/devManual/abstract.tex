\chapter{Introduction}
\label{s:introduction}
%
\section{About \NAME}
%
\NAME (Another QUAlity GPU-SPH) is a free CFD software licensed under GPLv3,
and developed by J.L. Cercos-Pita as part of his PhD at CEHINAV-UPM group
\footnote{Model Basin Research group, Technical University of Madrid (UPM),
\url{http://canal.etsin.upm.es}}. \NAME is based on Lagragian meshfree SPH
(Smoothed Particles Hydrodynamics) method.\rc
%
\NAME has been accelerated with OpenCL, that allows you to execute it over
CPU based platforms, over GPUs based platforms, and eventually, over every platform
developed in the future adapted to the OpenCL standard. For the moment the simultaneous
use of several platforms in the same simulation is not supported.\rc
%
The code has been widely validated, used and documented in several publications, for
instance \citet{Maciaetal_PTP_2012} where boundary integrals are tested or
\citet{perezrojas_cercos_stab12} where SPHERIC benchmark test case number 9 was simulated.
Some examples provided within the package have been studied in these publications.\rc
%
\NAME has been designed for UNIX like operative system, and tested on GNU/Debian Linux distributions.
%
For the moment no ports has been developed for Windows or Mac operative systems.
%
\NAME has been developed in C\texttt{++} and OpenCL\footnote{OpenCL is so quite similar to C}. Also Python extensions has been developed.
%
\section{License notes}
%
\NAME is free software: you can redistribute it and/or modify
it under the terms of the GNU General Public License as published by
the Free Software Foundation, either version 3 of the License, or
(at your option) any later version.\rc
%
\NAME is distributed in the hope that it will be useful,
but WITHOUT ANY WARRANTY; without even the implied warranty of
MERCHANTABILITY or FITNESS FOR A PARTICULAR PURPOSE.  See the
GNU General Public License for more details.\rc
%
You should have received a copy of the GNU General Public License
in 'LICENSE' file with \NAME package. If not, see
\url{http://www.gnu.org/licenses/}.
%
\section{Objectives of the present document}
%
This document has been created aiming to provide a tool that allows the \NAME developers to know:
%
\begin{enumerate}
	\item The structure of \NAME source code.
	\item The location of the source codes and helper files.
	\item The entry points to the main parts of AQUAgpusph
\end{enumerate}
%
This document is trying to complement the Doxygen documentation.
%
Such documentation can be get, either building it with CMake (see the user manual) or in the following web page (corresponding to the last stable \NAME package):\rc
%
\url{http://canal.etsin.upm.es/aquagpusph/doc/doxygen/stable}
%
\section{Required background}
%
The background required for the developers depends on the section of the code to be read/modified:
%
\begin{enumerate}
	\item The kernel of \NAME is written C\texttt{++}.
	\item The computational tools are written in OpenCL.
	\item The examples are built using XML files.
	\item Also Python extension may be used in some examples.
	\item The documentation (this document as well as the developers guide) is written in \LaTeX.
	\item The configuration and compilation is carried out with CMake.
\end{enumerate}

Also, depending on the modifications to be done, some background in CFD, and more specifically in SPH, may be required.
%